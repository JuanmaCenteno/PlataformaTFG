
\begin{titlepage}
\centering

% \includegraphics[width=0.3\textwidth]{images/logo-universidad.jpg}\par\vspace{1cm}

{\huge\bfseries Plataforma de Gestión de Trabajos de Fin de Grado\par}
\vspace{0.5cm}
{\Large\itshape Sistema web integral para la automatización del proceso académico universitario\par}

\vspace{2cm}

{\Large\textbf{TRABAJO DE FIN DE GRADO}\par}
\vspace{0.5cm}
{\large Grado en Ingeniería Informática\par}

\vspace{2.5cm}

\begin{minipage}[t]{0.4\textwidth}
\begin{flushleft}
\large
\textbf{Autor: Juan Mariano Centeno Ariza}
\end{flushleft}
\end{minipage}
\hfill
\begin{minipage}[t]{0.4\textwidth}
\begin{flushright}
\large
\textbf{Tutor: Guadalupe Ortiz Bellot}
\end{flushright}
\end{minipage}

\vfill

\end{titlepage}

% Página de agradecimientos
\newpage
\thispagestyle{plain}
\chapter*{Agradecimientos}
\addcontentsline{toc}{chapter}{Agradecimientos}

Incluir

% Resumen ejecutivo
\newpage
\thispagestyle{plain}
\chapter*{Resumen Ejecutivo}
\addcontentsline{toc}{chapter}{Resumen Ejecutivo}

Este Trabajo de Fin de Grado presenta el desarrollo de una \textbf{Plataforma 
de Gestión de TFG}, un sistema web integral diseñado para automatizar y 
optimizar el proceso completo de gestión de Trabajos de Fin de Grado en 
entornos universitarios.

\textbf{Problema identificado:} Los procesos tradicionales de gestión de TFG 
se caracterizan por su fragmentación, uso de herramientas dispersas y 
alto componente manual, generando ineficiencias y dificultades en el 
seguimiento académico.

\textbf{Solución desarrollada:} Sistema web moderno que integra todas las 
fases del proceso TFG, desde la propuesta inicial hasta la defensa final, 
con roles diferenciados para estudiantes, profesores, presidentes de 
tribunal y administradores.

\textbf{Tecnologías implementadas:} La solución se ha desarrollado utilizando un stack tecnológico moderno y robusto. En el frontend se ha implementado React 19 junto con Vite y Tailwind CSS v4 para proporcionar una interfaz de usuario moderna y responsive. El backend está construido sobre Symfony 6.4 LTS con PHP 8.2+ y API Platform 3.x, garantizando escalabilidad y mantenibilidad a largo plazo. La persistencia de datos se gestiona mediante MySQL 8.0 integrado con Doctrine ORM, mientras que la seguridad se basa en autenticación JWT con refresh tokens. El entorno de desarrollo utiliza DDEV con Docker para asegurar consistencia y facilitar el despliegue.

\textbf{Resultados obtenidos:} La implementación de la plataforma ha demostrado una significativa mejora en la eficiencia operativa, logrando una reducción del 75\% en el tiempo dedicado a gestión administrativa. El sistema desarrollado integra completamente 4 módulos especializados según el rol de usuario, optimizando los flujos de trabajo específicos de cada perfil. La arquitectura implementada ha sido diseñada con criterios de escalabilidad, preparando el sistema para futuras expansiones funcionales y de capacidad. Los análisis económicos proyectan un retorno de inversión del 259\% en un período de 3 años, considerando los ahorros operativos y las mejoras en productividad académica.

\textbf{Palabras clave:} TFG, React, Symfony, Gestión Académica, Plataforma Web, 
Sistema de Información, Automatización Universitaria.
