\documentclass[12pt,a4paper,oneside]{report}
\usepackage[utf8]{inputenc}
\usepackage[spanish,english]{babel}
\renewcommand{\chaptername}{Capítulo}
\usepackage[margin=2.5cm,top=3cm,bottom=3cm]{geometry}
\usepackage{fancyhdr}
\usepackage{graphicx}
\usepackage{hyperref}
\usepackage{listings}
\usepackage{xcolor}
\usepackage{tocloft}
\usepackage{titlesec}
\usepackage{longtable}
\usepackage{booktabs}
\usepackage{array}
\usepackage{multirow}
\usepackage{calc}
\usepackage{float}
\usepackage{setspace}
\usepackage{parskip}

% Definir tightlist si no existe
\providecommand{\tightlist}{%
  \setlength{\itemsep}{0pt}\setlength{\parskip}{0pt}}

% Definir pandocbounded si no existe
\providecommand{\pandocbounded}[1]{#1}

% Definir passthrough si no existe
\providecommand{\passthrough}[1]{\texttt{#1}}

% Configuración de spacing
\onehalfspacing
\setlength{\parskip}{6pt}

% Configuración de imágenes para ajuste automático
\usepackage{adjustbox}
\graphicspath{{processed/images/}}
\setkeys{Gin}{width=\linewidth,height=\textheight,keepaspectratio}

% Configuración de colores para código
\definecolor{codegreen}{rgb}{0,0.6,0}
\definecolor{codegray}{rgb}{0.5,0.5,0.5}
\definecolor{codepurple}{rgb}{0.58,0,0.82}
\definecolor{backcolour}{rgb}{0.95,0.95,0.92}

% Configuración de listings
\lstdefinestyle{mystyle}{
    backgroundcolor=\color{backcolour},   
    commentstyle=\color{codegreen},
    keywordstyle=\color{magenta},
    numberstyle=\tiny\color{codegray},
    stringstyle=\color{codepurple},
    basicstyle=\ttfamily\footnotesize,
    breakatwhitespace=false,         
    breaklines=true,                 
    captionpos=b,                    
    keepspaces=true,                 
    numbers=left,                    
    numbersep=5pt,                  
    showspaces=false,                
    showstringspaces=false,
    showtabs=false,                  
    tabsize=2
}
\lstset{style=mystyle}

% Headers y footers
\pagestyle{fancy}
\fancyhf{}
\fancyhead[R]{\small\leftmark}
\fancyfoot[C]{\thepage}

% Configurar marks para mostrar nombres de capítulos correctamente
\renewcommand{\chaptermark}[1]{\markboth{#1}{}}

% Configuración de títulos
\titleformat{\chapter}[hang]
{\normalfont\huge\bfseries}{\thechapter.}{1em}{}
\titlespacing*{\chapter}{0pt}{0pt}{20pt}

% Configuración de hyperlinks
\hypersetup{
    colorlinks=true,
    linkcolor=blue,
    filecolor=magenta,      
    urlcolor=cyan,
    citecolor=red,
    pdftitle={Plataforma de Gestión de TFG - Documentación Técnica},
    pdfauthor={Tu Nombre},
    pdfsubject={Trabajo de Fin de Grado - Ingeniería Informática},
    pdfcreator={Pandoc with LaTeX},
    pdfkeywords={TFG, React, Symfony, Plataforma Web}
}

% Configuración de tabla de contenidos
\setcounter{tocdepth}{3}
\setcounter{secnumdepth}{3}

% Redefinir el comando de inclusión de imágenes de Pandoc
\makeatletter
\def\maxwidth{\ifdim\Gin@nat@width>\linewidth\linewidth\else\Gin@nat@width\fi}
\def\maxheight{\ifdim\Gin@nat@height>0.8\textheight 0.8\textheight\else\Gin@nat@height\fi}
\makeatother
\setkeys{Gin}{width=\maxwidth,height=\maxheight,keepaspectratio}

\begin{document}

% Incluir portada

\begin{titlepage}
\centering

{\scshape\LARGE Universidad de Cádiz \par}
\vspace{1cm}
{\scshape\Large Escuela Superior de Ingeniería\par}
{\scshape\large Departamento de Informática\par}
\vspace{1.5cm}

\includegraphics[width=0.3\textwidth]{images/logo-universidad.jpg}\par\vspace{1cm}

{\huge\bfseries Plataforma de Gestión de Trabajos de Fin de Grado\par}
\vspace{0.5cm}
{\Large\itshape Sistema web integral para la automatización del proceso académico universitario\par}

\vspace{2cm}

{\Large\textbf{TRABAJO DE FIN DE GRADO}\par}
\vspace{0.5cm}
{\large Grado en Ingeniería Informática\par}

\vspace{2.5cm}

\begin{minipage}[t]{0.4\textwidth}
\begin{flushleft}
\large
\textbf{Autor: Juan Mariano Centeno Ariza}
\end{flushleft}
\end{minipage}
\hfill
\begin{minipage}[t]{0.4\textwidth}
\begin{flushright}
\large
\textbf{Tutor: Guadalupe Ortiz Bellot}\\
Dra. Guadalupe Ortiz Bellot\\
Departamento de Informática
\end{flushright}
\end{minipage}

\vspace{2cm}

{\large Curso Académico 2024-2025\par}
{\large Septiembre de 2025\par}

\vfill

\textit{``La tecnología es mejor cuando acerca a las personas.''}\\
\textit{-- Matt Mullenweg}

\end{titlepage}

% Página de derechos de autor
\newpage
\thispagestyle{empty}
\vspace*{\fill}
\begin{center}
\textcopyright\ 2025 Juan Mariano Centeno Ariza

\vspace{1cm}

Este documento ha sido elaborado siguiendo el estándar ISO/IEEE 16326:2009 \\
para documentación técnica de sistemas software.

\vspace{1cm}

\textbf{Plataforma de Gestión de TFG} \\
Sistema desarrollado con React 19, Symfony 6.4 LTS y MySQL 8.0

\vspace{1cm}

Trabajo presentado para la obtención del título de \\
\textbf{Graduado en Ingeniería Informática}

\end{center}
\vspace*{\fill}

% Página de agradecimientos
\newpage
\chapter*{Agradecimientos}
\addcontentsline{toc}{chapter}{Agradecimientos}

Quiero expresar mi sincero agradecimiento a todas las personas que han contribuido 
a la realización de este Trabajo de Fin de Grado:

A mi tutora, Dra. Guadalupe Ortiz Bellot, por su orientación experta, paciencia y apoyo 
continuo durante todo el proceso de desarrollo del proyecto.

A los profesores del Grado en Ingeniería Informática que han contribuido a mi 
formación técnica y académica.

A mi familia y amigos por su apoyo incondicional durante estos años de estudios.

A la comunidad de desarrolladores de código abierto cuyas herramientas y 
conocimientos han hecho posible este proyecto.

\vspace{2cm}
\begin{flushright}
Juan Mariano Centeno Ariza\\
Septiembre 2025
\end{flushright}

% Resumen ejecutivo
\newpage
\chapter*{Resumen Ejecutivo}
\addcontentsline{toc}{chapter}{Resumen Ejecutivo}

Este Trabajo de Fin de Grado presenta el desarrollo de una \textbf{Plataforma 
de Gestión de TFG}, un sistema web integral diseñado para automatizar y 
optimizar el proceso completo de gestión de Trabajos de Fin de Grado en 
entornos universitarios.

\textbf{Problema identificado:} Los procesos tradicionales de gestión de TFG 
se caracterizan por su fragmentación, uso de herramientas dispersas y 
alto componente manual, generando ineficiencias y dificultades en el 
seguimiento académico.

\textbf{Solución desarrollada:} Sistema web moderno que integra todas las 
fases del proceso TFG, desde la propuesta inicial hasta la defensa final, 
con roles diferenciados para estudiantes, profesores, presidentes de 
tribunal y administradores.

\textbf{Tecnologías implementadas:}
\begin{itemize}
    \item \textbf{Frontend:} React 19, Vite, Tailwind CSS v4
    \item \textbf{Backend:} Symfony 6.4 LTS, PHP 8.2+, API Platform 3.x
    \item \textbf{Base de datos:} MySQL 8.0 con Doctrine ORM
    \item \textbf{Autenticación:} JWT con refresh tokens
    \item \textbf{Desarrollo:} DDEV con Docker
\end{itemize}

\textbf{Resultados obtenidos:}
\begin{itemize}
    \item Reducción del 75\% en tiempo de gestión administrativa
    \item Sistema completo con 4 módulos diferenciados por rol
    \item Arquitectura escalable preparada para expansión
    \item ROI del 259\% proyectado en 3 años
\end{itemize}

\textbf{Palabras clave:} TFG, React, Symfony, Gestión Académica, Plataforma Web, 
Sistema de Información, Automatización Universitaria.

\newpage
\chapter*{Abstract}
\addcontentsline{toc}{chapter}{Abstract}

This Final Degree Project presents the development of a \textbf{TFG Management 
Platform}, a comprehensive web system designed to automate and optimize the 
complete process of managing Final Degree Projects in university environments.

\textbf{Identified Problem:} Traditional TFG management processes are 
characterized by fragmentation, use of scattered tools, and high manual 
component, generating inefficiencies and difficulties in academic tracking.

\textbf{Developed Solution:} Modern web system that integrates all TFG process 
phases, from initial proposal to final defense, with differentiated roles for 
students, professors, tribunal presidents, and administrators.

\textbf{Implemented Technologies:}
\begin{itemize}
    \item \textbf{Frontend:} React 19, Vite, Tailwind CSS v4
    \item \textbf{Backend:} Symfony 6.4 LTS, PHP 8.2+, API Platform 3.x
    \item \textbf{Database:} MySQL 8.0 with Doctrine ORM
    \item \textbf{Authentication:} JWT with refresh tokens
    \item \textbf{Development:} DDEV with Docker
\end{itemize}

\textbf{Results Obtained:}
\begin{itemize}
    \item 75\% reduction in administrative management time
    \item Complete system with 4 role-differentiated modules
    \item Scalable architecture prepared for expansion
    \item 259\% ROI projected over 3 years
\end{itemize}

\textbf{Keywords:} TFG, React, Symfony, Academic Management, Web Platform, 
Information System, University Automation.


% Tabla de contenidos
\renewcommand{\contentsname}{Índice}
\markboth{Índice}{Índice}
\tableofcontents
\newpage

% Lista de figuras
\renewcommand{\listfigurename}{Lista de Figuras}
\markboth{Lista de Figuras}{Lista de Figuras}
\listoffigures
\newpage

% Limpiar marks antes del contenido principal
\markboth{}{}

% Configurar numeración de capítulos
\setcounter{chapter}{0}

% Contenido principal
$body$

% Bibliografía (si existe)
\bibliographystyle{plain}
\bibliography{referencias}

\end{document}